\documentclass[a4paper, 12pt]{article}

% Корректность отображения всех шрифтов, кодировок и мат. символов
\usepackage[T2A]{fontenc}
\usepackage[utf8]{inputenc}
\usepackage[russian]{babel}
\usepackage{amssymb, amsmath, amsthm, mathtools}

\usepackage{lipsum}
% \usepackage{showframe}

% Уравнивание длин колонок на последней странице
% \usepackage{flushend}

% Настройка шрифта
\usepackage{fontspec}
\setmainfont{CMU Sans Serif}

% Отступ у первого абзаца
\usepackage{indentfirst}

% Всё, что нужно для tikz
\usepackage{tikz}
\usetikzlibrary{shapes, decorations, calc, arrows, angles}
\usetikzlibrary{3d, fit, backgrounds, decorations.text}
\usetikzlibrary{positioning, shapes.symbols}
\usetikzlibrary{decorations.pathreplacing, calligraphy}
\tikzset{>=latex}

% Римские числа из обычных
\newcommand{\rnum}[1]{\uppercase\expandafter{\romannumeral #1\relax}}

% Часто используемые обозначения
\newcommand{\N}{\mathbb{N}}
\newcommand{\Q}{\mathbb{Q}}
\newcommand{\R}{\mathbb{R}}
\renewcommand{\C}{\mathbb{C}}
\newcommand{\rk}{\mathrm{rk}}
\newcommand*{\bord}{\multicolumn{1}{c|}{}}

% Отступы
\usepackage[left=18mm, right=18mm, top=20mm, bottom=16mm, includefoot]{geometry}

% Пастельные цветовые рамки
\usepackage[many]{tcolorbox}

% Оболочка для теоремы
\newenvironment{theorem}[1][]{
    \smallskip\noindent{\bfseries Теорема.}\,
}{
    \par\smallskip
}

% Оболочка для доказательства
\renewenvironment{proof}{
    \smallskip\noindent\textit{Доказательство.}
}{
    \hfill$\blacksquare$\par\smallskip
}

% Оболочки для леммы, предложения и утверждения
\newcounter{lemma}
\newenvironment{lemma}[1][]{
    \refstepcounter{lemma}
    \smallskip\noindent{\bfseries Лемма \thelemma.}\,
}{
    \par\smallskip
}

% Для работы со списками
\usepackage{enumitem}

\begin{document}

\noindent\framebox{
    \begin{minipage}{\textwidth}
        \begin{center}
            В пространстве даны четыре попарно скрещивающиеся прямые. Докажите, что существует не более двух или бесконечно много прямых, пересекающих каждую из них.
        \end{center}
    \end{minipage}
}\bigskip

\begin{theorem}
    В пространстве даны три попарно скрещивающиеся прямые. Множество все прямых, компланарных с каждой из них, есть поверхность второго порядка.
\end{theorem}

Сначала докажем вспомогательное утверждение:

\begin{lemma}
    В пространстве даны три попарно скрещивающиеся прямые. Тогда существует прямая, компланарная с каждой из них.
\end{lemma}

\begin{proof}
    Пусть $\ell_1$, $\ell_2$ и $\ell_3$ --- данные прямые. Выберем точку $A$ на $\ell_2$ и проведём плоскость $\pi$, содержащую $\ell_1$ и эту точку. Если $\ell_3$ пересекает $\pi$ в точке $B$, то $AB$ --- искомая прямая, а если $\ell_3$ параллельна $\pi$, то через $A$ проведём прямую, параллельную $\ell_3$, она и будет искомой.
\end{proof}

Теперь докажем сформулированную выше теорему:

\begin{proof}
    Пусть $\ell_1$, $\ell_2$ и $\ell_3$ --- искомые прямые, причём
    $$
    \ell_i:
    \begin{cases}
        x = x_i + \alpha_it,\\
        y = y_i + \beta_it,\\
        z = z_i + \gamma_it.
    \end{cases}
    $$

    Выберем точку $M(x, y, z)$, не лежащую ни на одной из $\ell_i$, в пространстве и построим плоскости $\alpha_i$, содержащие $M$ и $\ell_i$ соответственно. Пересечение $\alpha_1 \cap \alpha_2 \cap \alpha_3$ непусто (в нём содержится хотя бы точка $M$), поэтому это либо точка, либо прямая. Если это прямая, то она компланарна каждой из $\ell_i$, их ГМТ мы хотим исследовать. Пересечение эти плоскостей является прямой тогда и только тогда, когда вектора нормалей этих плоскостей линейно зависимы. Это легко увидеть, записав систему уравнений. Пусть $\alpha_i: a_ix + b_iy + c_iz + d_i = 0$. Тогда пересечение задаётся системой
    $$
    \begin{cases}
        a_1x + b_1y + c_1z + d_1 = 0,\\
        a_2x + b_2y + c_2z + d_2 = 0,\\
        a_3x + b_3y + c_3z + d_3 = 0.
    \end{cases}
    $$

    Это система линейных уравнений 3 $\times$ 3. Известно, что она совместна (координаты точки $M$ удовлетворяют системе) и неопределена (решение --- прямая). Из курса алгебры, это выполняется тогда и только тогда, когда матрица коэффициентов вырождена, т.\,е.
    $$
    \det
    \begin{pmatrix}
        a_1 & b_1 & c_1\\
        a_2 & b_2 & c_2\\
        a_3 & b_3 & c_3
    \end{pmatrix} = 0.
    $$

    Найдём же вектора нормалей. В плоскости $\alpha_i$ гарантированно лежат два вектора: $(x - x_i, y - y_i, z - z_i)$ и $(\alpha_i, \beta_i, \gamma_i)$. Их векторное произведение и даст вектор нормали:
    $$
    \begin{array}{c}\displaystyle
        n_i = \det
        \begin{pmatrix}
            e_1 & e_2 & e_3\\
            x - x_i & y - y_i & z - z_i\\
            \alpha_i & \beta_i & \gamma_i
        \end{pmatrix} = {}\\\displaystyle
        \begin{pmatrix}
            \gamma_i(y - y_i) - \beta_i(z - z_i),& \alpha_i(z - z_i) - \gamma_i(x - x_i),& \beta_i(x - x_i) - \alpha_i(y - y_i)
        \end{pmatrix}.
    \end{array}
    $$

    Осталось составить из этих векторов матрицу и посчитать её определитель:

    $$
    \det
    \begin{pmatrix}
        \gamma_1(y - y_1) - \beta_1(z - z_1) & \alpha_1(z - z_1) - \gamma_1(x - x_1) & \beta_1(x - x_1) - \alpha_1(y - y_1)\\
        \gamma_2(y - y_2) - \beta_2(z - z_2) & \alpha_2(z - z_2) - \gamma_2(x - x_2) & \beta_2(x - x_2) - \alpha_2(y - y_2)\\
        \gamma_3(y - y_3) - \beta_3(z - z_3) & \alpha_3(z - z_3) - \gamma_3(x - x_3) & \beta_3(x - x_3) - \alpha_3(y - y_3)\\
    \end{pmatrix}
    $$

    Однако незачем считать его полностью. Нам важно лишь одно --- мы хотим получить поверхность второго порядка. Это может получиться в двух ситуациях --- либо все коэффициенты при третьих степенях сократятся, либо поверхность третьего порядка получится распадающаяся. Второй вариант очень сомнителен, потому что тогда мы получаем поверхность второго порядка и плоскость. А какую плоскость? Получается, три скрещивающиеся прямые (в общем виде) задают какую-то особенную плоскость. А это неправда, поэтому переключимся на сокращающиеся коэффициенты.

    Нам нужно посчитать только коэффициенты при мономах третьей степени и увидеть, что они все нулевые. Но опять же, ненужно перебирать все такие мономы. Достаточно посчитать только коэффициенты при $xyz$ и $x^2y$, остальные тоже зануляться из симметрии. Итак, вычислим коэффициент при $xyz$:
    $$
    \begin{array}{c}\displaystyle
        \alpha_2\beta_3\gamma_2 - \alpha_3\beta_1\gamma_2 + \alpha_3\beta_1\gamma_2 - \alpha_1\beta_2\gamma_3 + \alpha_1\beta_2\gamma_3 - \alpha_2\beta_3\gamma_2 - {}\\\displaystyle (\alpha_2\beta_1\gamma_3 - \alpha_1\beta_3\gamma_2 + \alpha_1\beta_3\gamma_2 - \alpha_3\beta_2\gamma_1 + \alpha_3\beta_2\gamma_1 - \alpha_2\beta_1\gamma_3) = 0.
    \end{array}
    $$
    и при $x^2y$:
    $$
    -\beta_3\gamma_1\gamma_2 - \beta_2\gamma_1\gamma_3 - \beta_1\gamma_2\gamma_3 - (-\beta_1\gamma_2\gamma_3 - \beta_3\gamma_1\gamma_2 - \beta_2\gamma_1\gamma_3) = 0.
    $$

    Значит, искомое ГМТ --- поверхность не более чем второго порядка. Докажем теперь, что именно второго. Если все коэффициенты при мономах второй степени сокращаться, получим плоскость. Если все такие прямые образуют плоскость, то и данные прямые лежат в одной плоскости, что неверно. Если же получается уравнение нулевого порядка, то это либо всё пространство, либо пустое множество. Пустого множества быть не может, т.\,к. в пространстве всегда найдётся прямая, пересекающая три попарно скрещивающиеся (лемма 1), а всё пространтво не может получиться, т.\,к. тогда любая прямая компланарна одной из $\ell_i$, а это не так (достаточно взять любую прямую, скрещивающуюся с $\ell_1$).
\end{proof}

Теперь приведём утверждение из лекционного курса (без доказательства):

\begin{lemma}
    У однополостного гиперболоида и гиперболического параболоида есть два семейства образующих. Через каждую точку поверхности проходит ровно одна образующая из каждого семейства. Образующие из разных семейств у гиперболического параболоида пересекаются, а у однополостного гиперболоида --- пересекаются или параллельны. Различные образующие из одного семейства скрещиваются.
\end{lemma}

Выберем из четырёх прямых три и построим поверхность второго порядка над ними. Эта поверхность --- либо однополостный гиперболоид, либо гиперболический параболоид (образующие есть только у них и у конуса, но у конуса они все проходят через одну точку). Данные нам прямые --- образующие полученной поверхности из одного семейства (назовём его первым), а прямые, ГМТ которых мы искали, образующие этой поверхности из другого семейства (назовём его вторым). Попадаем в один из двух случаев --- либо четвёртая прямая лежит в этой поверхности (т.\,е. тоже является её образующей), либо не лежит. 

Во втором случае четвёртая прямая пересекает поверхность не более чем в двух точках. Образующие из второго семейства через эти точки и могут быть искомыми прямыми. Поэтому в этом случае их не больше двух.

В первом случае, если наша поверхность --- это гиперболический параболоид, то прямых, пересекающих все четыре образующих, бесконечное количество, т.\,к. каждая из образующих второго семейства пересекает данные образующие первого семейства. Если же наша поверхность --- однополостный гиперболоид, то некоторые образующие из второго семейства окажутся параллельны. Но сколько их будет? Ответ даёт следующая

\begin{lemma}
    Для каждой образующей однополостного гиперболоида из одного семейства существует единственная образующая из второго семейства, параллельная ей.
\end{lemma}

\begin{proof}
    Одна существует --- она симметрична данной относительно центра. Назовём её $\ell$. Пусть существует и вторая $\ell^\ast$. Тогда (отношение параллельности транизитивно) $\ell^\ast \parallel \ell$. Но они из одного семейства, поэтому скрещиваются. Противоречие.
\end{proof}

То есть для каждой из трёх начальных образующих существует единственная образующая из второго семейства, параллельная им. Значит, проводя образующие второго семейства через точки четвёртой прямой, из бесконечно числа нужно будет исключить не более трёх прямых, останется бесконечное число.

\end{document}



