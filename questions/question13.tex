\section{Связка плоскостей в пространстве. Условие принадлежности плоскости связке, определённой тремя плоскостями}

\begin{definition}
    \textbf{Собственная связка плоскостей в пространстве} --- семейство плоскостей, проходящих через некоторую фиксированную точку.
\end{definition}

\begin{definition}
    \textbf{Несобственная связка плоскостей в пространстве} --- семейство плоскостей, параллельных некоторому фиксированному направлению.
\end{definition}

Аналогично рассуждению для пучка прямых, связка однозначно определяется тремя плоскостями и пишется как линейная комбинация левых частей уравнений этих плоскостей. И аналогично же получается

\begin{statement}
    Плоскости $\pi_i: A_ix + B_iy + C_iz + D_i = 0$, $i = 1, 2, 3, 4$ лежат в одной связке тогда и только тогда, когда
    $$\det
    \begin{pmatrix}
        A_1 & B_1 & C_1 & D_1\\
        A_2 & B_2 & C_2 & D_2\\
        A_3 & B_3 & C_3 & D_3\\
        A_4 & B_4 & C_4 & D_4
    \end{pmatrix} = 0.
    $$
\end{statement}


