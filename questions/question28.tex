\section{Эллипс и его геометрические свойства}

\begin{definition}[Геометрическое определение эллипса]
    \textbf{Эллипс} --- геометрическое место точек $X$, сумма расстояний от которых до некоторых фиксированных точек $F_1$ и $F_2$ постоянна и больше $|F_1F_2|$:
    $$
    |XF_1| + |XF_2| = 2a.
    $$
    Точки $F_1$ и $F_2$ называются \textbf{фокусами} эллипса.
\end{definition}

\begin{statement}
    Аналитическое и геометрическое определения эллипса эквивалентны.
\end{statement}

\begin{theorem}[Оптическое свойство]
    Касательная в точке $X$ эллипса является внешней биссектрисой угла $\angle F_1XF_2$.
\end{theorem}

\begin{theorem}[Изогональное свойство]
    Пусть дана точка $P$ снаружи эллипса. Проведём касательные $PA$ и $PB$ к эллипсу из неё. Тогда $PF_1$ и $PF_2$ изогональны относительно угла $\angle APB$.
\end{theorem}




