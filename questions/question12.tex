\section{Пучок плоскостей в пространстве. Условие принадлежности плоскости пучку, определённому двумя плоскостями}

\begin{definition}
    \textbf{Собственый пучок плоскостей в пространстве} --- семейство всех плоскостей, содержащих некоторую прямую.
\end{definition}

\begin{definition}
    \textbf{Несобственный пучок плоскостей в пространстве} --- семейство всех плоскостей, параллельных некоторой фиксированной плоскости.
\end{definition}

Опять же, для любых двух плоскостей однозначно определён пучок, в котором они лежат (т.\,к. однозначно определена прямая пересечения или плоскость, которой они параллельны).

\begin{statement}
    Плоскости $\pi_1$, $\pi_2$, $\pi_3$ ($\pi_i: \underbrace{A_ix + B_iy + C_iz + D_i}_{{} = f_i} = 0$) лежат в одном пучке тогда и только тогда, когда
    $$\rk
    \begin{pmatrix}
        A_1 & B_1 & C_1 & D_1\\
        A_2 & B_2 & C_2 & D_2\\
        A_3 & B_3 & C_3 & D_3
    \end{pmatrix} < 3
    $$
\end{statement}

\begin{proof}
    Заметим, что 
    $$\rk
    \begin{pmatrix}
        A_1 & B_1 & C_1 & D_1\\
        A_2 & B_2 & C_2 & D_2\\
        A_3 & B_3 & C_3 & D_3
    \end{pmatrix} < 3 \Leftrightarrow \exists\!\:(\alpha, \beta, \gamma) \ne (0, 0, 0):\alpha f_1 + \beta f_2 + \gamma f_3 = 0.
    $$
    Не ограничивая общности, пусть $\gamma \ne 0$. Тогда $\displaystyle f_3 = \frac{-\alpha f_1 - \beta f_2}{\gamma}$. 
\end{proof}

\begin{remark}
    Как следствие получаем, что пучок через две плоскости $\pi_1: f_1 = 0$ и $\pi_2: f_2 = 0$ пишется как
    $\lambda f_1 + \mu f_2 = 0$. Можно было сначала понять это и доказывать, как для пучка прямых на плоскости.
\end{remark}


