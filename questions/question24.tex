\section{Инварианты кривой второй степени}

\begin{definition}
    Многочлен $\chi = \det(X - \lambda E)$ называется \textbf{характеристическим многочленом} матрицы $X$.
\end{definition}

\begin{theorem}
    Значения характеристического многочлена матрицы квадратичной формы не меняются в зависимости от выбора ортогональной системы координат.
\end{theorem}

\begin{proof}
    Пусть в одной системе координат характеристический многочлен $\chi = \det(Q - \lambda E)$, а в другой (в которую мы перешли с помощью матрицы $C$) --- $\chi^\ast = \det(Q^\ast - \lambda E)$. Отметим, что в силу ортогональности матрицы $C$ имеем $E = C^TC$, а в силу леммы 22.1 $Q^\ast = C^TQC$. Итак,
    $$
    \chi^\ast(\lambda) = \det(Q^\ast - \lambda E) = \det(C^TQC - \lambda C^TC) = \det C^T\det(Q - \lambda E)\det C = (\det C)^2\chi(\lambda) = \chi(\lambda).
    $$
\end{proof}

\begin{orangebox}
    Квадрат определителя ортогональной матрицы равен $1$ в силу того, что
    $$
    (\det C)^2 = \det C \cdot \det C^T = \det(C \cdot C^{-1}) = \det E = 1.
    $$
    Как следствие последней теоремы получаем, что коэффициенты характеристического уравнения инварианты при ортогональных заменах. Посмотрим же на эти коэффициенты.
\end{orangebox}

А мы их уже видели! В доказательстве леммы 22.1 мы уже раскрывали скобки и получали, что коэффициент при $\lambda^2$ равен $1$ (не очень удивительно, что он инвариант, правда?), при $\lambda$ он равен $\tr Q$ со знаком <<минус>> (но красоты ради мы знак уберём), а свободный член равен $\det Q$. Мы нашли уже 2 инварианта, найдём и третий. Это определитель матрицы
$$
\mathcal{A} =
\begin{pmatrix}
    a_{11} & a_{12} & a_1\\
    a_{12} & a_{22} & a_2\\
    a_1 & a_2 & a_0
\end{pmatrix}.
$$
Дело в том, что при ортогональных заменах эта матрица меняется так же, как и матрица квадратичной части (настолько так же, что там и доказательство такое же). Единственное отличие в том, что вместо матрицы $C$ берётся \textbf{расширенная матрица преобразования}
$$
D \vcentcolon = 
\begin{pmatrix}
    \multicolumn{2}{c}{\multirow{2}{*}{$C$}} & x_0\\
    \multicolumn{2}{c}{} & y_0\\
    0 & 0 & 1
\end{pmatrix}
$$

Она уже, конечно, не ортогональная, но разложив её определитель по третьей строке, получим $\det D = \det C = \pm 1$. Итак,
$$
\det \mathcal{A}^\ast = \det(D^T\mathcal{A}D) = (\det D)^2 \det \mathcal{A} = \det \mathcal{A}.
$$

\begin{definition}
    Величины $I_1 \vcentcolon = \tr Q$, $I_2 \vcentcolon = \det Q$ и $I_3 \vcentcolon = \det A$ называются \textbf{ортогональными инвариантами} уравнения второго порядка.
\end{definition}

\begin{orangebox}
    \textbf{Очень важное замечание}. Как говорил Александр Александрович, <<Называть эти величины инвариантами кривой нельзя, их надо называть инвариантами уравнения>>. Действительно, домножим уравнение $F(x, y) = 0$ на ненулевое число $\lambda$. Кривая, задаваемая им, не изменится, а инварианты --- да, но предсказуемо: $I_k \mapsto \lambda^k I_k$. Потому что изменилось уравнение.
\end{orangebox}

\begin{theorem}[Из Веселова и Троицкого]
    Произвольный ортогональный инвариант $J$ многочлена второй степени, полиномиально зависящий от его коэффициентов, является многочленом от $I_1$, $I_2$, $I_3$.
\end{theorem}

Доказательство --- позже.

