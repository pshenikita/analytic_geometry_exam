\section{Сопряженные диаметры кривой второй степени. Касательные к кривой второй степени}

\begin{definition}
    Ненулевой вектор $(\alpha, \beta)$ имеет \textbf{асимптоническое направление} по отношению к кривой второго порядка, если
    $$
    \begin{pmatrix}
        \alpha & \beta
    \end{pmatrix}
    Q
    \begin{pmatrix}
        \alpha\\\beta
    \end{pmatrix} = 0,
    $$
    где $Q$ --- матрица квадратичной формы этой кривой.
\end{definition}

\begin{orangebox}
    Это свойство не меняется при умножении на ненулевой множитель, т.\,е. является свойством квадрики, а не уравнения.
\end{orangebox}

\begin{statement}
    Определение асимптотического направления корректно, т.\,е. не зависит от выбора системы координат.
\end{statement}

\begin{proof}
    Пусть $C$ --- матрица ортогонального преобразования. Тогда
    $$
    \begin{pmatrix}
        \alpha & \beta
    \end{pmatrix}
    Q
    \begin{pmatrix}
        \alpha\\\beta
    \end{pmatrix} = 
    \begin{pmatrix}
        \alpha\\\beta
    \end{pmatrix}^TQ
    \begin{pmatrix}
        \alpha\\\beta
    \end{pmatrix} = 
    \left(
        C\begin{pmatrix}
            \alpha^\ast\\\beta^\ast
        \end{pmatrix}
    \right)^TQC
    \begin{pmatrix}
        \alpha^\ast\\\beta^\ast
    \end{pmatrix} = 
    \begin{pmatrix}
        \alpha^\ast & \beta^\ast
    \end{pmatrix}\underbrace{C^TQC}_{Q^\ast}
    \begin{pmatrix}
        \alpha^\ast\\\beta^\ast
    \end{pmatrix}.
    $$
\end{proof}

\begin{theorem}
    Прямая $\ell$ неасимптотического направления по отношению к кривой второго порядка $\Gamma$ либо имеет с ней две точки пересечения (различные или совпавшие), либо не пересекается с ней. Прямая $\ell$ асимптотического направления по отношению к кривой второго порядка $\Gamma$ либо содержится в $\Gamma$, либо имеет с ней одну общую точку, либо не пересекается с ней.
\end{theorem}

\begin{proof}
    Рассмотрим пересечение параметрически заданной прямой
    $$
    \begin{cases}
        x = x_0 + \alpha t,\\
        y = y_0 + \beta t
    \end{cases}
    $$
    с данной кривой второго порядка. Подставив $x$ и $y$ в уравнение $F = 0$, получим уравнение $F(t) = 0$ со старшим коэффициентом, равным
    $$
    \begin{pmatrix}
        \alpha & \beta
    \end{pmatrix}
    Q
    \begin{pmatrix}
        \alpha\\\beta
    \end{pmatrix}.
    $$

    Если направление $(\alpha, \beta)$ неасимптотическое, то этот коэффициент не равен нулю и имеет $0$, $1$ или $2$ корня. Однако, если корень $1$, то в левой части просто выделился полный квадрат и два корня совпали. Если же направление $(\alpha, \beta)$ асимптотическое, то получаем линейное уравнение, которое имеет $0$, $1$ или бесконечно много корней, что соответствует перечисленным в утверждении теоремы случаям.
\end{proof}

\begin{theorem}
    Середины ходр кривой $\Gamma$ данного неассимптотического направления $(\alpha, \beta)$ лежат на прямой
    $$
    \alpha\cdot\frac{\partial F}{dx} + \beta\cdot\frac{\partial F}{dy} = 0.\eqno(\ast)
    $$
\end{theorem}

\begin{proof}
    Пусть имеем кривую второго порядка $\Gamma: F = 0$ и прямую $\ell$, заданную параметрически:
    $$
    \begin{cases}
        x = x_0 + \alpha t,\\
        y = y_0 + \beta t.
    \end{cases}
    $$

    Тогда, в силу неасимптотичности направления $(\alpha, \beta)$, уравнение $F(t) = 0$ квадратное, если оно пересекает $\Gamma$ в двух точках: $t_1$ и $t_2$ (выберем $(x_0, y_0)$ так, чтобы она была серединой отрезка, соединяющего эти две точки), то они находятся так
    $$
    F_2t^2 + F_1t + F_0 = 0,
    $$
    где
    $$
    F_2 = 
    \begin{pmatrix}
        \alpha & \beta
    \end{pmatrix}
    Q
    \begin{pmatrix}
        \alpha\\\beta
    \end{pmatrix},\quad F_1 = \alpha\cdot\frac{\partial F}{dx} + \beta\cdot\frac{\partial F}{dy},\quad F_0 = F(x_0, y_0).
    $$

    Заметим, что аффинные координаты точек пересечения $t_1$ и $t_2$ удовлетворяют уравнению
    $$
    \frac{t_1 + t_2}{2} = 0,
    $$
    т.\,к. точка $(x_0, y_0)$ --- середина отрезка между ними и имеет координату $0$ на $\ell$ (потому что мы её так выбрали). А из формул Виета, $t_1 + t_2 = -F_1 / F_2$. Мы знаем, что $F_2 \ne 0$, а поэтому $F_1 = 0$, а это и значит, что точка $(x_0, y_0)$ удовлетворяет уравнению $(\ast)$.

    Осталось показать, что уравнение $(\ast)$ действительно задаёт прямую (не вырождается в уравнение нулевой степени). Пусть это не так, т.\,е. оба коэффициента (при $x$ и при $y$) равны нулю:
    $$
    \begin{cases}
        a_{11}\alpha + a_{12}\beta = 0,\\
        a_{12}\alpha + a_{22}\beta
    \end{cases} \Rightarrow
    \begin{cases}
        a_{11}\alpha^2 + a_{12}\alpha\beta,\\
        a_{12}\alpha\beta + a_{22}\beta^2 = 0
    \end{cases}.
    $$

    Сложив эти уравнения, получим ровно условие
    $$
    \begin{pmatrix}
        \alpha & \beta
    \end{pmatrix}
    Q
    \begin{pmatrix}
        \alpha\\\beta
    \end{pmatrix},
    $$
    что противоречит неасимптотичности направления $(\alpha, \beta)$.
\end{proof}

\begin{definition}
    Прямая, существование которой мы доказали, называется \textbf{диаметром}, сопряжённым неасимптотическому направлению $(\alpha, \beta)$.
\end{definition}

\begin{orangebox}
    В окружности сопряжение --- это просто перпендикулярность.
\end{orangebox}

\begin{definition}
    \textbf{Центром кривой} $\Gamma$ называется такая точка $M$, что если $X$ лежит в $\Gamma$, то и $2M - X$ тоже лежит в $\Gamma$.
\end{definition}

\begin{orangebox}
    Центр может быть не один. А ещё, их может вообще не быть. Следующая таблица даёт представление о том, что происходит при каждом типе кривой:
    \begin{center}
        \begin{tabular}{| l | c |}
            \hline
            \textit{Эллипс} & \multirow{4}{*}{точка}\\
            \textit{Пара мнимых пересекающихся прямых} & \\
            \textit{Гипербола} & \\
            \textit{Пара пересекающихся прямых} & \\
            \hline
            \textit{Пара параллельных прямых} & \multirow{2}{*}{прямая}\\
            \textit{Пара совпавших прямых} & \\
            \hline
            \textit{Парабола} & нет\\
            \hline
        \end{tabular}
    \end{center}
\end{orangebox}

\begin{lemma}
    Пусть $M(x_0, y_0)$ --- центр кривой $\Gamma$. Существуют две различные прямые неасимптотических направлений, проходящие через $M$ и пересекающие $\Gamma$.
\end{lemma}

\begin{proof}
    Для точки и прямых утверждение очевидно, докажем его для коник. Возьмём две точки $P$ и $Q$, лежащие на $\Gamma$ и не симметричные относительно $M$ и пусть $P^\prime$ и $Q^\prime$ --- соответственно их образы при симметрии относительно $M$ Тогда $(PM \cap \Gamma) \supset\{P, P^\prime\}$, $(QM \cap \Gamma) \supset \{Q, Q^\prime\}$. Более того, третьих точек в персечениях нет, так как иначе прямая, содержащая их, имеет асимптотическое направление и содержится в $\Gamma$, т.\,е. $\Gamma$ распадается на две прямые и не является коникой.
\end{proof}

\begin{theorem}[Формула для нахождения центра]
    $M(x_0, y_0)$ является центром непустой кривой второго порядка $F = 0$ тогда и только тогда, когда
    $$
    \frac{\partial F}{dx} = \frac{\partial F}{dy} = 0.
    $$
\end{theorem}

\begin{proof}
    $\Rightarrow$. Пусть $(\alpha_i, \beta_i)$ ($i = 1, 2$) --- направляющие векторы прямых, определённых по предыдущей лемме. Точка точка $M(x_0, y_0)$, как середина соответствующих хорд, принадлежит соответствующим диаметрам, т.\,е. удовлетворяем уравнениям
    $$
    \alpha_i\cdot\frac{\partial F}{dx} = \beta_i\cdot\frac{\partial F}{dy} = 0.
    $$

    Это система уравнений относительно $\displaystyle\frac{\partial F}{dx}$ и $\displaystyle\frac{\partial F}{dy}$. Она совместна (т.\,к. однородна) и определена (т.\,к. матрица коэффициентов невырождена в силу неколлинеарности векторов $(\alpha_i, \beta_i)$). Значит, она имеет единственное решение
    $$
    \frac{\partial F}{dx} = \frac{\partial F}{dy} = 0.
    $$

    $\Leftarrow$. Пусть точка $M$ удовлетворяем выписанным в условии уравнениям. Тогда она лежит на любой прямой вида $(\ast)$, т.\,е. хотя бы на двух из них. Их пересечение --- с одной стороны --- центр, а с другой --- точка $M$. А пересечение двух непараллельных прямых --- это одна точка, поэтому $M$ --- центр.
\end{proof}

\begin{theorem}
    Если непустая кривая второго порядка имеет единственный центр, то диаметр, сопряжённый направлению $(\alpha, \beta)$, имеет неасимптотическое направление $(\alpha^\ast, \beta^\ast)$, причём диаметр, сопряжённый направлению $(\alpha^\ast, \beta^\ast)$, имеет направление $(\alpha, \beta)$.
\end{theorem}

\begin{proof}
    Рассмотрим сопряжённый к $(\alpha, \beta)$ диаметр
    $$
    \alpha\cdot\frac{\partial F}{dx} + \beta\cdot\frac{\partial F}{dy} = 0.
    $$

    Пусть эта прямая имеет направление $(\alpha^\ast, \beta^\ast)$. Тогда этот вектор должен при подстановке занулять линейную часть, т.\,е.
    $$
    \alpha(a_{11}\alpha^\ast + a_{12}\beta^\ast) + \beta(a_{12}\alpha^\ast + a_{22}\beta^\ast) = 
    \begin{pmatrix}
        \alpha & \beta
    \end{pmatrix}Q
    \begin{pmatrix}
        \alpha^\ast\\\beta^\ast
    \end{pmatrix}.\eqno(\star^1)
    $$

    Предположим, что направление $(\alpha^\ast, \beta^\ast)$ асимптотическое. Тогда выполнено ещё и 
    $$
    \begin{pmatrix}
        \alpha^\ast & \beta^\ast
    \end{pmatrix}Q
    \begin{pmatrix}
        \alpha^\ast\\\beta^\ast
    \end{pmatrix} = 0.\eqno(\star^2)
    $$

    Вместе $(\star^1)$ и $(\star^2)$ дают нам систему уравнений на $u = a_{11}\alpha^\ast + a_{12}\beta^\ast$ и $v = a_{12}\alpha^\ast + a_{22}\beta^\ast$. Матрица коэффициентов этой системы составлена из векторов $(\alpha, \beta)$ и $(\alpha^\ast, \beta^\ast)$. Она невырождена (иначе эти векторы коллинеарны, что точно неправда в силу $(\star^1)$ и неасимптотичности направления $(\alpha, \beta)$). А ещё эта система однородна, поэтому имеет единственное решение $u = v = 0$. Отсюда
    $$
    0 = u + v =
    \begin{pmatrix}
        \alpha & \beta
    \end{pmatrix}Q
    \begin{pmatrix}
        \alpha\\\beta
    \end{pmatrix}.
    $$

    Получили противоречие, значит, направление $(\alpha^\ast, \beta^\ast)$ неасимптотическое. Вторая часть теоремы получается из $(\star^1)$ путём транспонирования (матрица $Q$ симметрическая).
\end{proof}

\begin{orangebox}
    Из этой теоремы видно, что диаметры можно разбить на пары.
\end{orangebox}

\begin{definition}
    Два диаметра кривой с единственным центром, делящие пополам хорды, параллельные другому диаметру, называются \textbf{сопряжёнными}.
\end{definition}

\begin{definition}
    Точка $(x, y)$ алгебраической кривой с уравнением $F = 0$ называется \textbf{особой}, если в ней выполнены условия
    $$
    \frac{\partial F}{dx} = \frac{\partial F}{dy} = 0.
    $$
\end{definition}

\begin{definition}
    \textbf{Касательной} к алгебраической кривой $\Gamma$ в неособой точке $(x_0, y_0)$ называется прямая, проходящая через эту точку и пересекающая $\Gamma$ по совпавшим точкам \mbox{(или содержащаяся в ней)}.
\end{definition}

\begin{orangebox}
    Определение Александра Александровича: <<Касательная --- это кратный корень>>.
\end{orangebox}

\begin{theorem}
    Касательная к кривой $F = 0$ в неособой точке пишется так:
    $$
    \left.\frac{\partial F}{dx}\right|_{(x_0, y_0)}(x - x_0) + 
    \left.\frac{\partial F}{dy}\right|_{(x_0, y_0)}(y - y_0) = 0.
    $$
\end{theorem}

\begin{proof}
    Во-первых, она параллельная касательной (мы выше через градиент это показывали). А во-вторых, она проходит через правильную точку --- $(x_0, y_0)$. Значит, это и правда касательная! Если вы неудовлетворены этим доказательством, обратитесь к Веселову и Троицкому, там написано более длинное и убедительное рассуждение. Это доказательство нам рассказал Александр Александрович.
\end{proof}

\begin{theorem}[Критерий касания от Александра Александровича]
    Если $I_3 \ne 0$, то критерий того, что прямая $Ax + By + C = 0$ касается поверхности второго порядка $F = 0$ пишется так:
    $$
    \begin{pmatrix}
        A & B & C
    \end{pmatrix}\mathcal{A}
    \begin{pmatrix}
        A\\ B\\ C
    \end{pmatrix} = 0.
    $$
\end{theorem}

\begin{orangebox}
    Используем с умом! Может так получиться, что прямая, которую вы хотите проверить сюда подходит, но формально касательной не является. Как так? Очень просто. Она касается нашей кривой, но в бесконечно удалённой точке. Так, мы можем получить асимптоту у гиперболы: она сюда подходит, и в проективной геометрии она бы правда касалась нашей кривой. Однако в аффинной геометрии бесконечно удалённой точки нет, поэтому получается грусть и тоска. Чтобы проверить, что вы попали в этот случай, посмотрите, будет ли в вашей прямой лежать центр кривой $F = 0$. Если будет, значит, вы действительно попали в этот случай и такую плоскость мы касательной не считаем.

    Почему она вообще возникает? Потому что аффинная наука неправильная. Правильная --- проективная. В проективной геометрии всё хорошо, и найденая нами плоскость являлась бы касательной. В аффинной же возникает такая маленькая неприятность. Но всё равно штука очень полезная.

    И это не какой-то случай, которого вы никогда не встретите! Наш одногруппник попался на этом на контрольной, после чего мы и узнали, что есть такой подвох.
\end{orangebox}

\begin{proof}
    Из предыдущей теоремы можно сделать вывод, что касательная в точке $(x_0, y_0)$ пишется так:
    $$
    \begin{pmatrix}
        x_0 & y_0 & 1
    \end{pmatrix}\mathcal{A}
    \begin{pmatrix}
        x\\ y\\ 1
    \end{pmatrix} = 0.
    $$

    Кто не верит --- пусть проверит (просто раскройте скобки и убедитесь, что написана правда). Чтобы эта прямая совпала с прямой $Ax + By + C = 0$, коэффициенты у них должны быть пропорциональны. Иными словами,
    $$
    \lambda
    \begin{pmatrix}
        x_0 & y_0 & 1
    \end{pmatrix}\mathcal{A} = 
    \begin{pmatrix}
        A & B & C
    \end{pmatrix}.
    $$

    Матрица $\mathcal{A}$ симметрическая, т.\,е. $\mathcal{A}^T = \mathcal{A}$. А ещё, из условия $I_4 \ne 0$, поэтому кривая у $\mathcal{A}$ есть обратная. Транспонируем и умножаем на $(\lambda\mathcal{A})^{-1}$ слева.
    $$
    \frac{1}{\lambda}\mathcal{A}^{-1}
    \begin{pmatrix}
        A \\ B \\ C
    \end{pmatrix} = 
    \begin{pmatrix}
        x_0 \\ y_0 \\ 1
    \end{pmatrix}.
    $$

    Подставляем полученную строку в уравнение кривой:
    $$
    \begin{pmatrix}
        x_0 & y_0 & 1
    \end{pmatrix}\mathcal{A}
    \begin{pmatrix}
        x_0\\ y_0\\ 1
    \end{pmatrix} = 0,
    $$
    получим
    $$
    \left(
        \mathcal{A}^{-1}
        \begin{pmatrix}
            A\\ B\\ C
        \end{pmatrix}
    \right)^T\mathcal{A}\mathcal{A}^{-1}
    \begin{pmatrix}
        A\\ B\\ C
    \end{pmatrix} = 0 \Leftrightarrow 
    \begin{pmatrix}
        A & B & C
    \end{pmatrix}\mathcal{A}
    \begin{pmatrix}
        A\\ B\\ C
    \end{pmatrix} = 0.
    $$
\end{proof}


