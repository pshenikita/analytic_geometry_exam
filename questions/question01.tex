\section{Координаты на плоскости и в пространстве. Координаты точек и координаты векторов}

\begin{definition}
    Пусть $\{e_1, \ldots, e_k\}$ --- базис, а $v$ --- произвольный вектор. \textbf{Координатами вектора} $v$ называется набор чисел $(v_1, \ldots, v_k)$ такой, что $v = \sum_iv_ie_i$.
\end{definition}

Определения линейной зависимости, базиса и векторного пространства --- из курса алгебры.

\begin{statement}
    Определение выше корректно.
\end{statement}

\begin{proof}
    Требуется доказать, что координаты определены для каждого вектора в пространстве, причём, единственным образом:
    \begin{enumerate}
        \item \textbf{Существование}. Система $\{e_1, \ldots, e_k\}$ линейно зависима, а $\{e_1, \ldots, e_k, v\}$ --- нет. Значит, $\exists\!\:\lambda_1, \ldots, \lambda_k, \mu \in \R$, т.\,ч. $\lambda_1e_1 + \ldots + \lambda_ke_k + \mu v = 0$, при этом $\mu \ne 0$ (иначе уравнение становится неразрешимым в силу линейной независимости базиса). Поэтому разделим на $\mu$ и выразим вектор $v$:
            $$v = \sum_{i = 1}^k\left(-\frac{\lambda_i}{\mu}\right)e_i.$$
        \item \textbf{Единственность}. Пусть есть два выражения $v$ через базисные векторы:
            $$
                v = \alpha_1e_1 + \ldots + \alpha_ke_k,\quad v = \beta_1e_1 + \ldots + \beta_ke_k.
            $$

            Тогда, вычтя одно уравнение из другого, получим
            $$(\alpha_1 - \beta_1)e_1 + \ldots + (\alpha_k - \beta_k)e_k = 0.$$

            Это уравнение разрешимо только если $\alpha_i = \beta_i$ для всех $i$ (из линейной независимости базисных векторов). Значит, два наших выражения совпадают, а преполагалось, что они различны. Получили противоречие, доказывающее единственность координат каждого вектора.
    \end{enumerate}

    Получаем биекцию $\mathcal{F}$ из множества геометрических векторов размерности $k$ в $\R^k$. При этом, если $v \overset{\mathcal{F}}{\mapsto} (v_1, \ldots, v_k)$, то будем писать $v = (v_1, \ldots, v_k)$.
\end{proof}

Нетрудно заметить, что если $u = (u_1, \ldots, u_k)$ и $v = (v_1, \ldots, v_k)$, то $u + v = (u_1 + v_1, \ldots, u_k + v_k)$.

\begin{definition}
    Набор из точки $O$ и базисных векторов $\{e_1, \ldots, e_k\}$ называется \textbf{репером}. Точка $O$ при этом называется \textbf{началом координат}. Система координат, заданная таким образом, называется \textbf{аффинной}.
\end{definition}

\begin{definition}[Координаты точки]
    В аффинной системе координат $M \leftrightarrow (x_1, \ldots, x_k)$ --- координаты $\overrightarrow{OM}$ в базисе $\{e_1, \ldots, e_k\}$.
\end{definition}

Исходя из определения координат точки и замечания выше, с точки зрения координат корректна запись $\overrightarrow{AB} = B - A$.

