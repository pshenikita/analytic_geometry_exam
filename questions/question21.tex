\section{Плоские кривые второй степени. Аффинная классификация}

\begin{definition}
    Две квадрики \textbf{аффинно} (\textbf{метрически}) \textbf{эквивалентны}, если одна из них может быть переведена в другую аффинным (изометрическим) преобразованием.
\end{definition}

\begin{theorem}[Метрическая классификация квадрик]
    Две квадрики метрически эквивалентны тогда и только тогда, когда они имеют одинаковые канонический вид.
\end{theorem}

\begin{proof}
    $\Leftarrow$. Пусть есть две системы координат: $Oe_1e_2$ и $\widetilde{O}\widetilde{e}_1\widetilde{e_2}$, в которых кривая имеет одинаковые канонические уравнения. Тогда изометрия, переводящая один репер в другой, переводит одну кривую в другую.

    $\Rightarrow$. Рассмотрим две метрически эквивалентные квадрики и каноническую систему $Oxy$ для первой из них и её образ $\widetilde{O}\widetilde{x}\widetilde{y}$ при изометрии, переводящей первую квадрику во вторую (она существует из определения). В первой системе координат квадрики имеют уравнения $F_1(x, y) = 0$ и $F_2(x, y) = 0$, причём $F_1$ --- каноническое уравнение. Тогда вторая квадрика имеет два уравнения в новой системе координат: то же, что первая кривая имела в исходной системе, т.\,е. $F(x^\prime, y^\prime) = 0$, и то, что получается заменой координат, т.\,е. $F_2^\prime(x^\prime, y^\prime) = F_2(x(x^\prime, y^\prime), y(x^\prime, y^\prime)) = 0$. Таким образом, $F_1(x^\prime, y^\prime) = 0$ --- каноническое уравнение для второй квадрики (с точностью до умножения на множитель).
\end{proof}

\begin{lemma}
   Для любой кривой второго порядка существует аффинная система координат, в которой она имеет один из следующих видов:
    \begin{enumerate}
        \item $x^2 + y^2 = 1$ (эллипсы)
        \item $x^2 + y^2 = -1$ (мнимые эллипсы)
        \item $x^2 - y^2 = 1$ (гиперболы)
        \item $x^2 = 2y$ (параболы)
        \item $x^2 + y^2 = 0$ (пары мнимых пересекающихся прямых)
        \item $x^2 - y^2 = 0$ (пары пересекающихся прямых)
        \item $x^2 = 1$ (пары пареллельных прямых)
        \item $x^2 = -1$ (пары мнимых параллельных прямых)
        \item $x^2 = 0$ (пара совпавших прямых)
    \end{enumerate}
\end{lemma}

\begin{proof}
    Берём каноническое уравнение и <<растягиваем>> оси.
\end{proof}

\begin{theorem}[Аффинная классификация квадрик]
    Две квадрики аффинно эквивалентны тогда и только тогда, когда они имеют одинаковые названия.
\end{theorem}

\begin{proof}
    Аналогично теореме о метрической классификации получаем, что две квадрики с одинаковыми названиями аффинно эквивалентны.

    Обратно, докажем, что квадрики с разными названиями аффинно неэквивалентны. У коник никакие три точки не лежат на одной прямой, в отличие от остальных квадрик. Поскольку при аффинных преобразованиях сохраняется условие числа точек пересечения и деления в данном отношении, центр переходит в центр, а асимптотическое направление --- в асимптотическое. Так как у параболы нет центра, а у эллипса и гиперболы есть, причём у эллипса нет асимптотических направлений, а у гиперболы есть. Получаем, что эллипс, гипербола и парабола аффинно неэквивалентны. Пары прямых различаются геометрически. Наконец, у мнимого эллипса нет асимптотических направлений, а у пары мнимых параллельных прямых есть.
\end{proof}


