\section{Расстояние от точки до прямой на плоскости}

Здесь работаем в прямоугольной системе координат.

\begin{lemma}[О нормали к прямой]
    Пусть дана прямая $\ell: Ax + By + C = 0$. Тогда $n = (A, B) \perp \ell$.
\end{lemma}

\begin{proof}
    Если $v = (\alpha, \beta) \parallel \ell$, то $A\alpha + B\beta = 0$ и (т.\,к. система координат прямоугольна) $(n, v) = 0$. Значит, $v \perp n$.
\end{proof}

\begin{theorem}
    Пусть даны прямая $\ell: Ax + By + C = 0$ и произвольная точка $M = (x_0, y_0)$. Тогда
    $$
    \rho(M, \ell) = \frac{|Ax_0 + By_0 + C|}{\sqrt{A^2 + B^2}}.
    $$
\end{theorem}

Здесь следует ответить на несколько вопросов. Первый --- что мы называем расстоянием? Расстоянием мы называем наименьшую длину вектора от одного объекта до другого. Второй --- почему расстояние существует? Выберем на прямой репер с аффинной координатой $t$, тогда расстояние от фиксированной $M$ до точки на $\ell$ с координатой $t$ --- это функция $t \mapsto \rho(t)$. Причём на $\pm\infty$ эта функция бесконечно большая. А здесь математический анализ говорит, что у этой функции должен быть минимум. Вот этот минимум и есть расстояние от точки до прямой.

\begin{proof}
    Теперь в доказательстве ответим на третий вопрос --- почему расстояние от точки до прямой --- это перпендикуляр? Проведём из точки $M$ перпендикуляр и наклонную на $\ell$. Из элементарной геометрии мы знаем, что против большего угла лежит большая сторона, а значит, наклонная больше перпендикуляра.

    Выберем на прямой произвольную точку $O = (x_1, y_1)$ и проведём вектор $u = \overrightarrow{OM} = (x_0 - x_1, y_0 - y_1)$. Искомое расстояние --- длина проекции $u$ на нормаль $\ell$ --- вектор $n = (A, B)$. У для нахождения проекции вектора на вектор мы выводили формулу:
    $$
    u^\parallel = \frac{(u, n)}{(n, n)}n = \frac{A(x_0 - x_1) + B(y_0 - y_1)}{A^2 + B^2}n = \frac{Ax_0 + By_0 \overbrace{{} - Ax_1 - By_1}^{C}}{A^2 + B^2}n.
    $$
    Итак, $\displaystyle\rho(M, \ell) = |u^\parallel| = \frac{|Ax_0 + By_0 + C|}{\sqrt{A^2 + B^2}}$.
\end{proof}

