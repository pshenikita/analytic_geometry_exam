\section{Кривые второй степени, проходящие через пятерки и четверки точек}

\begin{theorem}
    Существует и единственна квадрика, проходящая через данные различные пять точек, никакие четыре из которых не лежат на одной прямой.
\end{theorem}

\begin{proof}
    Пусть $P_i(x_i, y_i)$ ($i = 1, \ldots, 5$) --- точки в некоторой прямоугольной системе координат. Для нахождения коэффициентов уравнения искомой квадрики возникает система из 5 линейных уравнений
    $$
    a_{11}x_i^2 + 2a_{12}x_iy_i + a_{22}y_i^2 + 2a_1x_i + 2a_2y_i + a_0 = 0,\quad i = 1, \ldots, 5
    $$
    от 6 неизвестных с точностью до умножения на ненулевой множитель. Такая система всегда имеет решение. Оно однозначно с точностью до умножения на константу, если уравнения линейно независимы. Допустим противное. Пусть, например, пятое уравнение является линейной комбинацией первый четырёх, так что любая квадрика, проходящая через $P_1, \ldots, P_4$, проходит и через $P_5$. Рассмотрим два случая:
    \begin{enumerate}
        \item Три точки из $P_1, \ldots, P_4$, например, $P_1$, $P_2$, $P_3$ лежат на одной прямой, которую мы обозначим $\ell$. Проведём прямую $m$, содержащую $P_4$ и не содержащую $P_5$. Так как 4 точки не лежат на одной прямой, получаем, что $m \ne \ell$ и $m \cup \ell$ --- квадрика, не содержащая $P_5$. Противоречие.
        \item Никакие три точки из $P_1, \ldots, P_4$ не лежат на одной прямой. Тогда определены две квадрики:
            $$
            q_1 \vcentcolon = (P_1P_2) \cup (P_3P_4)\quad\text{и}\quad q_2 \vcentcolon = (P_1P_4) \cup (P_2P_3).
            $$
            По предположению $P_5 \in q_1$, $P_5 \in q_2$. Но пересечение $q_1 \cap q_2$ равно $\{P_1, P_2, P_3, P_4\}$. Противоречие.
    \end{enumerate}

Полученные противоречия доказывают теорему.
\end{proof}


