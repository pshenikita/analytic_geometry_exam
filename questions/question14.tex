\section{Векторное произведение векторов. Определение и основные свойства. Вычисление векторного произведения в ортогональных координатах}

Сначала посмотрите следующий билет.

\begin{definition}
    \textbf{Векторным произведением} векторов $u$ и $v$ называется вектор $[u, v]$ такой\footnotemark, что $V(\ast, u, v) = (\ast, [u, v])$ в смысле равенства функций. 
\end{definition}

\footnotetext{Функция $V$ линейна (из предыдущего билета) и такой вектор есть (и только один) в силу теоремы 3.1}

\begin{lemma}[Свойства векторного произведения]
    \begin{enumerate}[nolistsep, noitemsep]
        \item $[u, v] \perp u, v$
        \item $\left|[u, v]\right| = |S(u, v)|$
        \item Если $u$ и $v$ линейно независимы, то базис $\{u, v, [u, v]\}$ положительно ориентирован
        \item Функция $f(u, v) = [u, v]$ полилинейна и кососимметрична
    \end{enumerate}
\end{lemma}

\begin{proof}
    Докажем по пунктам:
    \begin{enumerate}
        \item $(u, [u, v]) = V(u, u, v) = 0$, значит $u \perp [u, v]$. Аналогично для $v$.
        \item Возьмём вектор $w$, перпендикулярный $u$ и $v$. Тогда $[u, v] \parallel w$ и поэтому
            $$|S(u, v)| \cdot |w| = |V(u, v, w)| = |([u, v], w)| = |[u, v]|\cdot|w| \Rightarrow |w| = |S(u, v)|.$$
        \item Заметим, что $V(u, v, [u, v]) = ([u, v], [u, v]) > 0$ при $u, v \ne 0$ из положительной определённости скалярного произведения. Значит, $\{u, v, [u, v]\}$ --- положительно ориентированный базис
        \item Верно в силу полилинейности и кососимметричности функции $V(\ast, u, v)$.
    \end{enumerate}
\end{proof}

\begin{statement}[Формула для векторного произведение в прямоугольной положительно ориентированной системе координат]
    Пусть $u = (u_1, u_2, u_3)$, $v = (v_1, v_2, v_3)$. Тогда
    $$
    [u, v] = \det\begin{pmatrix}
        e_1 & e_2 & e_3\\
        u_1 & u_2 & u_3\\
        v_1 & v_2 & v_3
    \end{pmatrix} = 
    \left(\det
        \begin{pmatrix}
            u_2 & u_3\\
            v_2 & v_3
        \end{pmatrix},\det
        \begin{pmatrix}
            u_3 & u_1\\
            v_3 & v_1
        \end{pmatrix},\det
        \begin{pmatrix}
            u_1 & u_2\\
            v_1 & v_2
        \end{pmatrix}
    \right).
    $$
\end{statement}

\begin{proof}
    Возьмём вектор $w = (w_1, w_2, w_3)$. Посчитаем $V(u, v, w)$, разложив определитель по строке:
    $$\begin{array}{c}
        V(u, v, w) = \det\begin{pmatrix}
            u_1 & u_2 & u_3\\
            v_1 & v_2 & v_3\\
            w_1 & w_2 & w_3
        \end{pmatrix} = w_1\det
        \begin{pmatrix}
            u_2 & u_3\\
            v_2 & v_3
        \end{pmatrix} + w_2\det
        \begin{pmatrix}
            u_3 & u_1\\
            v_3 & v_1
        \end{pmatrix} + {}\\{} + w_3\det
        \begin{pmatrix}
            u_1 & u_2\\
            v_1 & v_2
        \end{pmatrix} = (w, [u, v]).
    \end{array}
    $$
\end{proof}

\begin{lemma}[Формула Лагранжа]
    $$
    [a, [b, c]] = b(a, c) - c(a, b).
    $$
\end{lemma}

\begin{proof}
    Обе части линейны по $a$, $b$ и $c$. Поэтому достаточно проверить равенство на базисных векторах (т.\,к. линейная функция задаётся своими значениями на базисных векторах). Заметим, что если $b = c$, то обе части обращаются в ноль. Поэтому осталось проверить при $b \ne c$. Если векторы $a$, $b$ и $c$ попарно различны, то $[b, c] = \pm a$, поэтому правая часть нулевая. Левая часть нулевая, т.\,к. зануляются оба скалярных произведения. Значит, нужно проверить при $a = b \ne c$. Также можно заметить, что функции в левой и правой части кососимметричны, поэтому достаточно рассмотреть только этот случай, причём порядок векторов $a$, $b$ и $c$ можно выбрать так, чтобы $[b, c] = a$ и $[b, a] = c$. Заметим, что при этом левая часть равна $c$, а правая --- тоже $c$. 
\end{proof}

\begin{theorem}[Тождество Якоби]
    $$
    [a, [b, c]] + [c, [a, b]] + [b, [c, a]] = 0.
    $$
\end{theorem}

\begin{proof}
    Из леммы 14.2 получаем
    $$
    [a, [b, c]] + [c, [a, b]] + [b, [c, a]] = b(a, c) - c(a, b) + a(c, b) - b(c, a) + a(b, c) - a(b, c) = 0.
    $$
    Требуемое получается из симметричности скалярного произведения.
\end{proof}


