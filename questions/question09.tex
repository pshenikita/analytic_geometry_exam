\section{Плоскость в пространстве. Параметрическое задание и задание уравнением}

Плоскость в пространстве (как и прямую на плоскости) можно задать, выбрав на неё репер:

$$
\begin{cases}
    x = x_0 + \alpha_1 t + \beta_1 s,\\
    y = y_0 + \alpha_2 t + \beta_2 s,\\
    z = z_0 + \alpha_3 t + \beta_3 s
\end{cases}\quad (t, s) \in \R^2.
$$

А ещё плоскость можно задать уравнением первого порядка: $Ax + By + Cz + D = 0$, где $(A, B, C) \ne (0, 0, 0)$.

\begin{statement}
    Эти два задания эквивалентны.
\end{statement}

\begin{proof}
    $\Rightarrow$. Легко видеть, что точка $M = (x_1, y_1, z_1)$ лежит в плоскости тогда и только тогда, когда 
    $$\det
    \begin{pmatrix}
        x_1 - x_2 & y_1 - y_2 & z_1 - z_2\\
        \alpha_1 & \alpha_2 & \alpha_3\\
        \beta_1 & \beta_2 & \beta_3
    \end{pmatrix} = 0.
    $$
    Действительно, это условие соответствую компланарности базисных векторов плоскости и вектора $\overrightarrow{OA}$, где $O$ --- начало координат. Разложив этот определитель по первой строке, получим уравнение первого порядка по $x$, $y$, $z$. Если все его коэффициенты нулевые, то базисные векторы сонаправлены, а такого быть не может.

    $\Leftarrow$. Аналогично случаю прямой на плоскости.
\end{proof}


