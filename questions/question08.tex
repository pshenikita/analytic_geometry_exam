\section{Пучок прямых на плоскости. Собственные и несобственые пучки}

\begin{definition}
    \textbf{Собственный пучок прямых на плоскости} --- семейство всех прямых, проходящих через некоторую фиксированную точку.
\end{definition}

\begin{definition}
    \textbf{Несобственный пучок прямых на плоскости} --- семейство всех прямых, параллельных некоторому фиксированному направлению.
\end{definition}

Нетрудно заметить, что для любых двух различных прямых плоскости $\ell_1$ и $\ell_2$ существует единственный пучок, содержащий их.

На самом деле, пучок через две прямые $\ell_1: A_1x + B_1y + C_1 = 0$ и $\ell_2: A_2x + B_2y + C_2 = 0$ пишется следующим образом:
$$\lambda(A_1x + B_1y + C_1) + \mu(A_2x + B_2y + C_2) = 0.$$

Если $\ell_1 \parallel \ell_2$, пучок получается несобственым (пропорциональность коэффициентов сохраняется), а иначе --- собственным (действительно, если есть точка $(x_0, y_0)$, обнуляющая оба выражения в левых частях уравнений прямых, она обнуляет и их линейную комбинацию). Отсюда можно сделать вывод, что прямая $\ell_3$ лежит в пучке, образованном прямыми $\ell_1$ и $\ell_2$ тогда и только тогда, когда вектор $(A_3, B_3, C_3)$ есть линейная комбинация векторов $(A_1, B_1, C_1)$ и $(A_2, B_2, C_2)$. Переформулировка даёт следующее

\begin{statement}
    Прямые $\ell_1$, $\ell_2$ и $\ell_3$ ($\ell_i: A_ix + B_iy + C_i = 0$) лежат в одном пучке тогда и только тогда, когда
    $$\det
    \begin{pmatrix}
        A_1 & B_1 & C_1\\
        A_2 & B_2 & C_2\\
        A_3 & B_3 & C_3
    \end{pmatrix} = 0
    $$
\end{statement}

